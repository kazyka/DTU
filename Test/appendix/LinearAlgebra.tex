\chapter{Linear Algebra}

A matrix $\matr{A}$ has elements $A_{ij}$ where $i$ indexes the rows, and $j$ indexes the columns. We use $\matr{I}_N$ to denote the $N \times N$ identity matrix (also called the unit matrix) and where there is no ambiguity over dimensionality we simply use $\matr{I}$. The transpose matrix $\matr{A}^T$ has elements $(\matr{A}^T)_{ij} = \matr{A}_{ji}$. From the definition of transpose, we have

\textcolor[rgb]{0.80,0.00,0.02}{Show example}

\begin{equation}\label{eq:transposedefinition}
  (\matr{A}\matr{B})^T = \matr{B}^T\matr{A}^T
\end{equation}

which can be verified by writing out the indices. The inverse of $\matr{A}$ denoted $\matr{A}^{-1}$ satisfies

\begin{equation}\label{eq:inversedefinition}
  \matr{A}\matr{A}^{-1} = \matr{A}^{-1}\matr{A} = \matr{I}
\end{equation}

Because $\matr{A}\matr{B}\matr{B}^{-1}\matr{A}^{-1} = \matr{I}$, we have

\begin{equation}\label{eq:inverseparantesedefinition}
  (\matr{A}\matr{A})^{-1} = \matr{B}^{-1}\matr{A}^{-1}
\end{equation}

Also we have

\begin{equation}\label{eq:inversetranposedefinition}
  (\matr{A}^T)^{-1} = (\matr{A}^{-1})^{T}
\end{equation}

